\documentclass[a4paper]{article}

\usepackage{microtype}
\usepackage{fullpage}
\usepackage{mathtools}
\usepackage{amsfonts}
\usepackage{tikz}
\usepackage{pgfplots}
\usepackage{hyperref}
\usepackage[nameinlink]{cleveref}
\usepackage{amsthm}
\usepackage{xcolor}
\usepackage[backend=biber,style=authoryear,maxbibnames=99,hyperref=true]{biblatex} 

\addbibresource{src/paper/references.bib}

\usepgfplotslibrary{fillbetween}
\usetikzlibrary{shapes}
\pgfplotsset{compat=1.17}

\newtheorem{proposition}{Proposition}
\newtheorem{corollary}{Corollary}
\newtheorem{lemma}{Lemma}
\newtheorem{assumption}{Assumption}

\newcommand{\dt}{\mathrm{d}t}
\newcommand{\ds}{\mathrm{d}s}
\newcommand{\di}{\mathrm{d}i}
\newcommand{\E}{\mathbb{E}}

\title{Hybrid platforms and bargaining power}
\author{Martin Stancsics}

\begin{document}

\maketitle

\begin{abstract}
\end{abstract}


\section{Motivation}


\section{Model}

This section introduces the model used throughout the paper.
I use the terms ``platform'' and ``fringe sellers'', and present my ideas in the context of an intermediated market, such as an online marketplace or an application store.
However, the model is more general, and the results apply to a broader class of settings.
Chief among these is an upstream-downstream setting, with a single upstream producer and a continuum of downstream sellers.

Assume that there are two types of players in the market: a platform $P$ and a continuum of fringe sellers $F_i$.


\subsection{Production}

I make use of a reduced-form, but quite general approach to model production and demand.
First, I assume that all fringe firms are ex-ante identical.
This implies that the total profits of the platform and the fringe taken together have a simple functional form $f(n_P, n_F)$.
\begin{assumption}
    \label{ass:identical_fringe}
    The total profits of the platform and the fringe are described by some function $f: \mathbb{R}^+_0 \times \mathbb{R}^+_0 \to \mathbb{R}^+_0$.
\end{assumption}
Here, $n_F$ is the mass of fringe sellers, and $n_P$ measures the degree to which the platform sells its own products.
For example, in a setting with differentiated products and monopolistic competition, $n_F$ and $n_P$ represents the product variety of the fringe and the platform, respectively.
In case the platform operates as a pure intermediary, $n_P = 0$.

I will assume that total profits are increasing in both arguments.
\begin{assumption}
    \label{ass:monotone_profits}
    $f(n_P, n_F)$ is increasing in both $n_P$ and $n_F$.
\end{assumption}
Such profit functions arise in a variety of settings in which the profit reduction from increased competition is dominated by extra sales due to increased variety.
One such example is \textcite{anderson2020aggregative}, where the demand exhibits love for variety, while consumers have access to an outside option.
Alternatively, indirect network effects, such as in two-sided markets, can also result in such profit functions.


\subsection{Profit sharing}

Let us assume that the fringe firms are unable to sell their products directly to consumers.
Instead, they must sell through the platform.
For this service, the platform may charge a fee, and obtain a share of the fringe's profits.

Let us assume that the platform and the fringe shares their total profits according to the following rule.
\begin{assumption}
    \label{ass:profit_sharing}
    Let $w: [0, 1] \to \mathbb{R}_0^+$ such that $\int_0^1 w(s) \ds = 1$. 
    \begin{align*}
        \pi_P(N_P, N_F) &= \int_0^1 w(s) f(N_P, s N_F) \ds, \\
        \pi_F(N_P, N_F) &= f(N_P, N_F) - \int_0^1 w(s) f(N_P, s N_F) \ds.
    \end{align*}
\end{assumption}
While at first it might seem rather arbitrary, and much more convoluted than a simple percentage split, it has quite a few nice properties compared to the latter.
Furthermore, \cref{sec:profit_sharing} provides some cooperative and non-cooperative microfoundations for this rule.
Let us start by examining the various parts of this rule, and then discuss how they impact the platform's profit share.

The main feature of this rule is that the platform's bargaining power depends on the substitutability of the fringe firms.
The following proposition demonstrates this idea.
\begin{proposition}
    \label{prop:outcome_based_bargaining_power}
    Let $f, \tilde{f}: \mathbb{R}^+_0 \times \mathbb{R}^+_0 \to \mathbb{R}^+_0$ two different total profit functions such that $f(N_P, N_F) = \tilde{f}(N_P, N_F)$ for some $N_P, N_F$ and $f(n_P, n_F) \leq \tilde{f}(n_P, n_F)$ for all $n_P < N_P, n_F < N_F$.
    Furthermore, let us denote the corresponding platform profit shares by $\pi_P$ and $\tilde{\pi}_P$.
    
    Then, $\pi_P \leq \tilde{\pi}_P$.
\end{proposition}
In words, when comparing two different total profit functions (i.e., demand systems, production functions), even if the total profits are the same under some pair $N_P, N_F$, profit shares might differ.
In particular, the platform's profits are higher when it needs fewer firms to achieve a certain total profit level (i.e., when the fringe firms are more substitutable).
\textcolor{red}{GRAPH HERE}
This coincides with the intuitive idea that the platform's bargaining power is higher when it does not really mind loosing a few fringe sellers.

The other main ingredient of the sharing rule, $w(s)$, captures the platform's \emph{innate} bargaining power.
It is independent of the production function, demand, or the fringe firms' substitutability, and might represent, for example, that the platform and the fringe firms have different power to propose allocations during the (not explicitly modeled) negotiation process.\footnote{
    For a more thorough exploration of this idea, see \cref{app:profit_sharing} and \textcite{hart1996bargaining}.
}
One way to think about it is that the platform's share is randomly drawn from the total profit according to some distribution, and $w$ is the probability density function of this distribution.
When the distribution $W_1$ first order stochastically dominates another distribution $W_2$, the platform is more likely to draw a value from with a higher mass of fringe firms, and therefore obtain a higher profit share in case of $W_1$ than in the case of $W_2$.
The next proposition formalizes this idea.
\begin{proposition}
    \label{prop:innate_bargaining_power}
    Let $w, \tilde{w}: [0, 1] \to \mathbb{R}_0^+$ be two different weight functions such that $\int_0^x w_1(s) \ds \leq \int_0^x w_2(s) \ds$ for all $x \in [0, 1]$.
    As before, let us denote the corresponding platform profit shares by $\pi_P$ and $\tilde{\pi}_P$.
    
    Then, $\pi_P(N_P, N_F) \leq \tilde{\pi}_P(N_P, N_F) \; \forall\, N_P \geq 0, N_F > 0$.
    Furthermore, this inequality is strict if $\int_0^x w_1(s) \ds < \int_0^x w_2(s) \ds$ on some non-zero measure set.
\end{proposition}


\subsection{Entry and equilibrium}

Let us now turn to the determination of the equilibrium number of fringe firms.
Assume that there is an unlimited number of potential fringe entrants, indexed by $i \in \mathbb{R}^+_0$.
In the first stage of the game, each fringe firm must create a viable product at the investment cost $I_F > 0$.
In the second stage, firms that made this investment plus the platform engage in producing and selling their respective products.
If there are $N_F$ fringe entrants and $N_P$ platform products, then, as discussed previously, the total profit they make together is given by $f(N_P, N_F)$.
Finally, the platform and the entrants share this profit according to the profit sharing rule from \cref{ass:profit_sharing}.

The equilibrium number of entrants is determined by a free entry condition: in the end, the fringe firms' profits should be equal to the investment cost.
\begin{assumption}
    \label{ass:free_entry}
    In equilibrium, $\pi_F(N_P, N_F) = I_F N_F$.
\end{assumption}

In order to guarantee a unique equilibrium, let us make the following, additional assumption about the profit and bargaining weight functions.
\begin{assumption}
    \label{ass:single_crossing}
    Let $f$ and $w$ be such that the following holds for all $N_F, N_P \geq 0$:
    \begin{gather*}
        f'_{n_F} (N_P, N_F) < \int_0^1 s w(s) f'_{n_F} (N_P, s N_F) \ds \\
        \text{or} \\
        f''_{n_F} (N_P, N_F) < \int_0^1 s^2 w(s) f''_{n_F} (N_P, s N_F) \ds.
    \end{gather*}
\end{assumption}
\textcolor{red}{TODO: Find interpretable sufficient conditions for this assumption.}


\section{Results}

Some preliminary (hand-wavily proven) results about equilibrium profits and entry.

\begin{proposition}
    \label{prop:share_of_platform}
    Assume that $f, w$ are continuously differentiable and that $w$ has a non-zero measure support.
    Then $\pi_P$ is also differentiable and
    $$\frac{\partial \pi_P(N_P, n_F)}{\partial N_P} > 0$$.
\end{proposition}

\begin{proposition}
    \label{prop:share_of_fringe}
    Assume that $f, w$ are twice continuously differentiable and that $w$ has a non-zero measure support.
    Then $\pi_F$ is also differentiable. Let $N_F > 0$.
    \begin{align*}
        &\text{If } \frac{\partial^2 f(N_P, n_F)}{\partial n_P \partial n_F} > 0 \;\forall n_F \leq N_F, \text{ then } \frac{\partial \pi_F(N_P, N_F)}{\partial N_P} < 0, \\
        &\text{if } \frac{\partial^2 f(N_P, N_F)}{\partial n_P \partial n_F} > 0 \;\forall n_F \leq N_F, \text{ then } \frac{\partial \pi_F(N_P, N_F)}{\partial N_P} > 0
    \end{align*}
    for all $N_P \geq 0$.
\end{proposition}

\begin{corollary}
    \label{fringe_entry}
    Assume that $f, w$ are twice continuously differentiable and that $w$ has a non-zero measure support.
    Let $N_F^*$ denote the equilibrium number of fringe firms.
    \begin{align*}
        &\text{If } \frac{\partial^2 f(n_P, n_F)}{\partial n_P \partial n_F} > 0 \forall n_P \leq N_P, \text{ then } \frac{\partial N_F^*}{\partial N_P} < 0. \\
        &\text{If } \frac{\partial^2 f(n_P, n_F)}{\partial n_P \partial n_F} > 0 \forall n_P \leq N_P, \text{ then } \frac{\partial N_F^*}{\partial N_P} > 0.
    \end{align*}
\end{corollary}


\section{Example}
\label{sec:example}

This section presents an example application of the model, based on \parencite[]{anderson2021hybrid}.
The setting is a (possibly hybrid) platform market with monopolistically competing fringe firms.

\subsection{Demand}

Imagine that there is a unit mass of consumers, looking to buy one product each.
They choose from a continuum of products, indexed by $i$.
Customer $j$ derives the following utility from buying product $i$:
\begin{align*}
    u^F_{ij} = v^F_i - p^F_i + \mu\epsilon^F_{ij},
\end{align*}
where $v^F_i$ is the value of product $i$, $p^F_i$ is its price, and $\epsilon^F_{ij}$ is an idiosyncratic taste shock.
Throughout this paper I assume that the value of each product is the same: $v^F_i = v^F$.
This is to simplify the analysis, but is not crucial for getting the results.  % TODO: look into relaxing this assumption

Additionally, customers can also choose from a unit mass of outside options, yielding utility $u^0_ij = \mu\epsilon^0_{ij}$. $\epsilon^F_{ij}$ and $\epsilon^0_{ij}$ are assumed to be independent and identically distributed (i.i.d.) across consumers and products, and follow a standardized Type I Extreme Value distribution. This distributional assumption, along with the fact that each consumer consumes only one product, will lead to a tractable, logit-form demand function.

\subsection{Production}

Each (horizontally differentiated) product is produced by a single, monopolistically competitive (fringe) seller.
The production entails a constant marginal cost $c^F_i$.
As with the value, I assume that the marginal cost is the same for all products: $c^F_i = c^F$.
Facing the demand described in the previous paragraphs, the sellers choose their price $p^F_i$ to maximize profits.

Additionally, the platform itself can also offer mass $N_P$ of its own products.
The value and marginal cost of these products are denoted by $v^P$ and $c^P$, respectively.
Furthermore, consumers have the same i.i.d. taste shocks ($\epsilon^P_{ij}$) for platform products as for the products offered by the other sellers.
Finally, I assume that the product prices its products as if they were produced by separate, monopolistically competitive sellers.
That is, it does not take into account the fact that it can affect a non-zero measure of the prices, and therefore the aggregate demand.\footnote{
    As I show in section TODO the appendix, this assumption is more innocuous than it seems.
    If the platform can commit to the price of its products before the entry of the fringe firms, then the results are the same regardless if the platform behaves as a continuum of atomic firms or one single firm with many products.
    If the platform cannot make this commitment, and determines its prices after entry decisions are made, the results change quantitatively, but the main ideas carry over.  % TODO: check this!!!!!!!!!!
}

\subsection{Intermediation}

Assume that the sellers and the consumers can only interact through an intermediary platform, denoted by $P$.
Without the platform, the fringe firms are not able to sell their products, and thus make no profits.
For this service, the platform can charge a lump-sum entry fee $F_F$ to each fringe firm.\footnote{
    This is a critical difference from \textcite[]{anderson2021hybrid}, where the platform charges a per-unit royalty.
    % TODO: elaborate on this
}
In the baseline model, I assume that entry is free for the consumers, and they are always present on the platform, no matter the other players' entry and pricing decisions.\footnote{
    This essentially means that I assume a one-sided platform, without the usual strong network effects.
    % TODO: see if I can relax this assumption
}

As mentioned above, instead of operating as a pure marketplace, the platform can also produce and offer its own products.
In case it does, it is called a hybrid platform.
Additionally, the platform can choose to exclude all the fringe firms from the market (e.g. by setting a sufficiently high entry fee), and only offer its own products.  % TODO: should I allow it in the baseline case, or only after a policy intervention?

\subsection{Entry and timing}

There is an infinite measure of fringe firms, who are deciding whether to enter the market.
Entering has two, separate costs: an exogenous investment cost $I_F$, and the platform entry fee $F_F$.
One can conceptualize the first as usual fixed costs, such as the cost of setting up production, or designing a product.
Meanwhile, the second is a payment for using the platform's services.

In the benchmark model, the game starts by the platform announcing its entry fee $F_F$.
In the bargaining case, the platform cannot commit to entry fees, so this step is skipped.
Next, each fringe firm decides whether to create a product at cost $I_F$ and enter the market.
In the bargaining model, the entry fee is decided at this point, as a result of a negotiation between the platform and the firms that have made the investment.
After that, the firms that have made the investment decide if they also want to enter the platform for the announced fee $F_F$.\footnote{
    In the benchmark model, separating these two decisions is redundant, as, in any subgame perfect equilibrium, any firm that makes the investment will also enter the platform.
    However, in the model with bargaining, this distinction is crucial.
}
Finally, the platform and the fringe firms simultaneously choose the prices for their products.
Each consumer then chooses the one maximizing their utility, and profits are realized.

\subsection{Bargaining}

The main difference between the benchmark and the bargaining model is in determining the entry fee $F_F$.
In the former, the platform unilaterally sets the fee, and the fringe treats it as a take it or leave it offer.
On the other hand, in the bargaining model, I assume that the entry fee is the result of a negotiation between the parties.
% TODO: entry fee cap?

Instead of modeling the bargaining process in detail, choose a reduced form approach. I assume that the final, net profits (after paying and receiving entry fees) correspond to the Shapley value of the participants in the following, cooperative game.\footnote{
    This way of modeling bargaining has precedents in the industrial organization literature \parencite[e.g.][]{hart1990property,segal2003collusion,inderst2003bargaining,montez2007downstream}.
    Furthermore, a number of papers show that various extensive-form bargaining games result in this outcome \parencite{gul1989bargaining,winter1994demand,hart1996bargaining,inderst2003bargaining,stole1996intra}.
}
Assume that the value of a coalition consisting of the platform and a measure $N_F$ of fringe firms is the total profits that they can achieve by selling their products in the last stage \emph{using monopolistic pricing}.\footnote{
    Note that lump-sum entry fees do not influence this value, as they are just transfers between coalition members.
}
The value of a coalition without a platform is simply zero.
% TODO: elaborate on how entry fees are not too important.


\section{Equilibrium}

\subsection{Producer profits}


As shown in \textcite[]{anderson2021hybrid}, the utility functions described in \cref{sec:model} give rise to a logit-type demand function.
Specifically, the demand for product $i$ of producer $T \in \{P, F\}$ is given by:
\begin{align*}
    x_{Ti} = \frac{\exp\left( \frac{v_T - p_{Ti}}{\mu} \right)}{A}
\end{align*}
where
\begin{align}
    A = \int_0^{N_F} \exp\left( \frac{v_F - p_{Fi}}{\mu} \right) \di + \int_0^{N_P} \exp\left( \frac{v_P - p_{Pi}}{\mu} \right) \di + 1.
    \label{eq:aggregate}
\end{align}
The producers' variable profit function is then given by
\begin{align}
    \pi^v_{Ti} = ( p_{Ti} - c_T ) \frac{\exp\left( \frac{v_T - p_{Ti}}{\mu} \right)}{A}.
    \label{eq:variable_profit}
\end{align}

Notice that, as each producer is infinitesimal, its pricing decision does not affect the aggregate $A$.
It follows that the profit maximizing price is a simple additive markup over marginal cost
\begin{align*}
    p^*_{Ti} = c_T + \mu,
\end{align*}
and the optimal profit is
\begin{align}
    \pi^{v*}_{Ti} = \mu \frac{\exp \left( \frac{v_T - c_T - \mu}{\mu} \right)}{A}.
    \label{eq:optimal_profit}
\end{align}
For fringe firms, their total profit (including the investment cost $I_F$ and the entry fee $F_F$) is given by
\begin{align}
    \pi^{t*}_F = \mu \frac{\exp \left( \frac{v_F - c_F - \mu}{\mu} \right)}{A} - I_F - F_F.
    \label{eq:fringe_profit}
\end{align}

For ease of notation, let us define the following:
\begin{align*}
    V_T = \exp \left( \frac{v_T - c_T - \mu}{\mu} \right).
\end{align*}
Then, equilibrium per-product demand and variable profit can be expressed as $V_T/ A$ and $\mu V_T/ A$, respectively, and the total aggregate is simply
\begin{align*}
    A = N_P V_P + N_F V_F + 1.
\end{align*}

\subsection{The benchmark model}

Recall that there is an infinity of potential fringe entrants looking to enter the market.
Therefore, total profits in equilibrium must be zero.
Let us for the moment assume that fixed fees are low enough for some fringe firms to enter the market.
In this case, setting $\pi^{t*}_F$, equation \eqref{eq:fringe_profit} implies that the size of the aggregate is
\begin{align}
    A = \mu \frac{V_F}{F_F + I_F}.
    \label{eq:aggregate_eq}
\end{align}
Substituting this into equation \eqref{eq:aggregate} and rearranging gives the number of fringe firms that enter the market:
\begin{align*}
    N_F = \frac{\mu}{F_F + I_F} - N_P \frac{V_P}{V_F} - \frac{1}{V_F}.
\end{align*}

If, on the other hand, entry costs are too high, no fringe firms enter the market ($N_F = 0$).
In that case, the aggregate demand is just
\begin{align*}
    A = N_P V_P + 1.
\end{align*}

In the first period of the game, the platform chooses an entry fee to maximize its total profit, consisting of revenue from its own sales and the collected entry fees.
First, let us examine the case when the fees are low enough for the fringe to be feasible.
Profits from sales can be obtained by combining equations \eqref{eq:variable_profit} and \eqref{eq:aggregate_eq}:
\begin{align*}
    \pi_P^{v} = N_P (F_F + I_F) \frac{V_P}{V_F}.
\end{align*}
% TODO: other cases
It is increasing in the (1) number of the platform's products, (2) the platform's product advantage (larger market share) and (3) the entry costs for fringe firms (less competition).
The other component of the platform's profit, entry fees collected from the fringe, are given by
\begin{align*}
    \pi_P^{f} = N_F F_F = F_F \left[ \frac{\mu}{F_F + i} - N_P \frac{V_P}{V_F} - \frac{1}{V_F} \right].
\end{align*}

Combining these two, the platform's total profit as a function of the entry fee can be expressed as
\begin{align}
    \pi_P^{t} = \mu\frac{F_F}{F_F + I_F} + N_P I_F \frac{V_P}{V_F} - \frac{F_F}{V_F}
    \label{eq:hybrid_profit}
\end{align}
Note, that even though the platform's profit is increasing in the number of its products, the optimal entry fee does not depend on it.
In particular, the optimal entry fee is given by
\begin{align*}
    F_F^* = \sqrt{\mu I_F V_F} - I_F.
\end{align*}
As a consequence, as long as the fringe is feasible at the optimal entry fee, the size of the aggregate, and hence consumer welfare is also unaffected by the platform operating in hybrid mode.
Substituting it into the profit function, the maximized profit is
\begin{align*}
    \pi_P^{t} = \mu - 2\sqrt{\frac{I_F \mu}{V_F}} + \frac{I_F}{V_F} (N_P V_P + 1).
\end{align*}

Now let us consider the case when the fringe is not present in the market.
In this case, the collected entry fees are obviously zero, and the platform's profit is simply given by
\begin{align*}
    \pi_P^{t} = \pi_P^{v} = \mu \frac{ N_P V_P}{N_P V_P + 1}.
\end{align*}
In this regime, total aggregate, and hence consumer welfare are increasing in the number of the platform's products.
This is mechanical: more variety leads to higher welfare.

Finally, note that the profit of the platform in the hybrid regime is higher than in the retail regime, when the former is feasible.
This is a consequence of the fact that equation \eqref{eq:hybrid_profit} is strictly concave in $F_F$, therefore $F_F^*$ is a global maximum.
It implies that there is no jump in no point where the fringe would be feasible under optimal entry fees, but the platform would prefer to operate in retail mode.
Therefore, in the benchmark model, a platform becomes a hybrid platform, or increasing the number of its own products, always has a weakly positive effect on the aggregate, and thus consumer welfare.
This is in contrast to \textcite{anderson2021hybrid}, where a platform operating in hybrid mode sets a higher entry fee to advantage its own products, which in turn leads to a smaller aggregate.

% TODO: intuition
% Figure

\subsection{Bargaining outcomes}

Now consider the case when instead of the platform setting take it or leave it entry fees, it must negotiate with the fringe about the division of profits.
In this case, the timing is slightly different, as the platform cannot commit to an entry fee in the first period.
Instead, the game starts by fringe firms' investment decisions.
After that, in the second period, bargaining takes place between the platform and the fringe firms that have entered the market.


The participants negotiate over the aggregate profits that they can achieve in the subsequent period (assuming non-collusive, monopolistic pricing, as before).
I assume that bargaining outcomes are the Shapley values of the participants, where the value of a coalition is defined as the total profits that the coalition can achieve by selling their products.
As the platform is necessary to reach the customers, the value of any coalition that does not include the platform is zero.
Furthermore, using equation \eqref{eq:variable_profit}, the value of a coalition with $n$ fringe firms and the platform is
\begin{align*}
    v(n) = \mu \frac{n V_F + N_P V_P}{n V_F + N_P V_P + 1}.
\end{align*}

\Cref{prop:one_sided} gives a simple formula for the Shapley value of the platform in case $N_F$ fringe firms made the initial investment:
\begin{align*}
    \pi^t_P &= \varphi_P = \int_0^1 v(tN_F) \dt \\
            &= \mu \left[ 1 - \frac{\log \left(1 + \frac{N_F V_F}{N_P V_P + 1} \right)}{N_F V_F} \right]
\end{align*}
Subsequently, the fringe gets the remainder of the profits, therefore
\begin{align}
    \pi^v_F &= \varphi_F =  1 - \int_0^1 v(tN_F) \dt \\
            &= \mu \left[ \frac{\log \left( 1 + \frac{N_F V_F}{N_P V_P + 1} \right)}{N_F V_F} - \frac{1}{N_P V_P + N_F V_F + 1} \right]
            \label{eq:fringe_variable_profit}
\end{align}
and the total profits of the whole fringe are
\begin{align*}
    \pi^t_F = \mu \left[ \frac{\log \left( 1 + \frac{N_F V_F}{N_P V_P + 1} \right)}{N_F V_F} - \frac{1}{N_P V_P + N_F V_F + 1} \right] - I_F N_F
\end{align*}
The free entry condition implies that fringe profits are zero, therefore the number of fringe entrants is given by the implicit equation
\begin{align}
    \mu \left[ \frac{\log \left( 1 + \frac{N_F V_F}{N_P V_P + 1} \right)}{N_F V_F} - \frac{1}{N_P V_P + N_F V_F + 1} \right] - I_F N_F = 0
    \label{eq:free_entry_bargaining}
\end{align}

\textcolor{red}{
    What I really want to show at this point is that if $N_F$ is determined by equation \eqref{eq:free_entry_bargaining}, then the total aggregate is decreasing in $N_P$.
    Unfortunately, I don't have a closed form for $N_F^*$, and the IFT does not seem to lead to useful results, either.
    Therefore, at this point, all I can do is show it through some numerical examples.
    That won't guarantee of course that it holds in general.
}

\textcolor{red}{
    I have a graphical reasoning in mind though, that I might be able to make precise in the future.
    For the moment, I'd rather not include it as it might be completely wrong.
}

%\section{Extensions}

%\section{Conclusion}


\appendix

\printbibliography


\section{Shapley value in the platform game}

Consider a cooperative game with two types of players: a major player $P$ and $n$ smaller players.
I.e., $N = \{P, F_1, \dots, F_n\}$.
Assume that (1) no coalition of players can achieve a positive value without the participation of $P$ and (2) players $F_i$ are identical.
Let $n_T(S)$ denote the number of players of type $T \in \{P, F\}$ in coalition $S$. Then, the value that any coalition $S \subset N$ can achieve is the following:
\begin{align*}
    v(S) = \begin{cases}
        0                              & \text{if } P \notin S \\
        f\left(\frac{n_F(S)}{n}\right) & \text{otherwise}.
    \end{cases}
\end{align*}
This coalition-form game corresponds to the setting in this paper with a (one-sided) platform and a number of potential entrants.

I start by characterizing the case when the game is superadditive to provide some support for using the Shapley value as he bargaining outcome.\footnote{For example, the results in \textcite{gul1989bargaining} rely on superadditivity, while monotonicity suffices for \textcite[]{hart1996bargaining}.}

\begin{proposition}
    \label{prop:monotone}
    The game $(N, v)$ is monotone and superadditive if and only if $f$ is increasing and $f(0) \geq 0$. % It is supermodular if and only if $f$ is convex.
\end{proposition}

\begin{proof}[Proof% of Proposition \ref{prop:monotone}
    ]
    Monotonicity is evident. For superadditivity, note that for any coalitions $S_1, s_2$ such that $S_1 \cap s_2 = \emptyset$, $P \notin S_1$ or $P \notin s_2$. WLOG assume it is the latter, therefore $v(S_2) = 0$. As a result, $v(S_1) + v(S_2) = v(S_1) \leq v(S_1 \cup S_2)$ holds if and only if $(N, v)$ is monotone.
\end{proof}

Now let us look at the limit of the Shapley-values as the number of fringe firms goes to infinity.\footnote{
    As this game is of bounded variation, this result can also be obtained in a more direct way using the main theorem from \textcite{fogelman1980asymptotic}.
    However, I believe the following proof is instructive, and drives home the idea that the continuous approximation is rather accurate even in the case of a relatively small number of players.
}
I use the following notation: $\varphi_P^n$ denotes the Shapley-value of player $P$ if there exist $n$ players of type $F$.
The next proposition shows that the Shapley value of the platform can be described by a simple integral.

\begin{proposition}
    \label{prop:one_sided}
    Let $f$ be continuous on [0, 1]. Then
    \begin{align*}
        \varphi_P^\infty = \lim_{n \to \infty} \varphi_p^n = \int_0^1 f(t) \dt .
    \end{align*}
\end{proposition}

\begin{proof}[Proof% of Proposition \ref{prop:one_sided}
    ]
    Let $R$ denote a permutation of the set of players ($N$). Additionally, let us denote the players preceding $i$ by $\mathcal{P}_i^R$. The value of player $P$ is their expected marginal contribution averaged over all permutations of $N$:
    \begin{align*}
        \varphi_P^n = \frac{1}{(n+1)!} \sum_R v(\mathcal{P}_P^R \cup \{i\}) - v(\mathcal{P}_P^R)
    \end{align*}
    First, note that $v(\mathcal{P}_P^R) = 0$ for any permutation, as no coalition can achieve a positive value without player $P$. Furthermore, using the fact that all agents of type $A$ are identical implies that $v(\mathcal{P}_P^R \cup \{i\})$ only depends on the number of agents in the coalition. More precisely, 
    \begin{align*}
        v(\mathcal{P}_P^R \cup \{i\}) = f(n_A(\mathcal{P}_P^R \cup \{i\}) / n) = f(|\mathcal{P}_P^R| / n).
    \end{align*}
    Finally, the set of permutations in which $k$ number of players precede $P$ is independent of $n$, i.e.
    \begin{align*}
        \{R \mid |\mathcal{P}_P^R| = k\} = n! \quad \forall\, k.
    \end{align*}
    Putting all the above together, the value of player $P$ can be expressed as
    \begin{align*}
        \varphi_P^n &= \frac{1}{(n+1)!} \sum_{k=0}^n n! f(k / n) \\
        &= \frac{1}{n+1} \sum_{k=0}^n f(k / n) \\
        &= \frac{n}{n+1} \underbrace{\frac{1}{n} \sum_{k=0}^{n-1} f(k / n)}_{=S_n} + \frac{1}{n+1} f(1).
    \end{align*}
    $S_n$ are just the left Riemann-sums of function $f$ on the interval $[0, 1]$. Therefore, if $f$ is continuous (and thus Riemann-integrable), then $S_n \to \int_0^1 f(t)$, and thus
    \begin{align*}
        \lim_{n \to \infty} \varphi_P^n &= \lim_{n \to \infty} \frac{1}{n+1} \sum_{k=1}^n f(k / n) \\
        &= \lim_{n \to \infty}\underbrace{\frac{n}{n+1}}_{\to 1} \frac{1}{n} \sum_{k=0}^{n-1} f(k / n) + \underbrace{\frac{1}{n+1} f(1)}_{\to 0} \\
        &= \int_0^1 f(t) \dt .
    \end{align*}
\end{proof}

Finally, the aggregated value of the fringe can be obtained using the efficiency of the Shapley value.

\begin{corollary}
    \label{cor:fringe_value}
    The aggregated Shapley-value of the fringe is
    \begin{align*}
        \varphi_F^\infty = f(1) - \int_0^1 f(t) \dt.
    \end{align*}
    Furthermore, if $f$ is differentiable on $[0, 1]$, Then the value of the fringe can also be expressed as
    \begin{align*}
        \varphi_F^\infty = \int_0^1 t f'(t) \dt.
    \end{align*}
\end{corollary}

\begin{proof}[Proof% of Corollary \ref{cor:fringe_value}
    ]
    The first equality comes from the efficiency of the Shapley-value. The values of all players sum up to $f(1)$ for all $n \in \mathbb{N}$, therefore
    \begin{align*}
        \lim_{n \to \infty} \sum_{i=1}^n \varphi_{F_i}^n = \lim_{n \to \infty} (1 - \varphi_P^n ) = 1 - \int_0^1 f(t).
    \end{align*}
    The second can be obtained by integration by parts:
    \begin{align*}
        \int_0^1 t f'(t) \dt = tf(t) \mid_0^1 - \int_0^1 f(t) \dt = f(1) - \int_0^1 f(t) \dt
    \end{align*}
\end{proof}


\section{Miscellaneous lemmas and proofs}

The following proposition shows that, if the platform is operating in a more hybrid mode (meaning it has more or better products), then, for a given aggregate size, the fringe obtains a lower share of the profits.
This is due to the increased bargaining power of the platform.

\begin{proposition}
    \label{prop:fringe_profit_decreasing_in_N_P}
    For a fixed aggregate size $A$, fringe profit is decreasing in $N_P$ and $V_P$.
\end{proposition}
\begin{proof}
    Let $N_P' V_P' > N_P V_P$.
    Furthermore, let $N_F V_F = A - N_P V_P - 1$ and $N_F' V_F' = A - N_P' V_P' - 1$ for some $A > 1$.
    Finally, let
    \begin{align*}
        v(n) &= \mu \frac{n V_F + N_P V_P}{n V_F + N_P V_P + 1}, \\
        v'(n) &= \mu \frac{n V_F + N_P' V_P'}{n V_F + N_P' V_P' + 1}.
    \end{align*}

    Observe that for any $t \in [0, 1)$,
    \begin{align*}
        v(tN_F) &= \frac{t N_F V_F + N_P V_P}{t N_F V_F + N_P V_P + 1} \\
                &= \frac{tA - t + (1 - t) N_P V_P}{tA - t + (1 - t) N_P V_P + 1} \\
                &< \frac{tA - t + (1 - t) N_P' V_P'}{tA - t + (1 - t) N_P' V_P' + 1} \\
                &= \frac{t N_F' V_F' + N_P' V_P'}{t N_F' V_F + N_P V_P + 1} \\
                &= v'(tN_F').
    \end{align*}
    This in turn implies that
    \begin{align*}
        \pi_F^v = \int_0^1 v(t) \dt < \int_0^1 v'(t) \dt = (\pi_F^v)',
    \end{align*}
    which is what we wanted to show.

\end{proof}


\begin{lemma}
    \label{lem:shape_of_pi_F}
    For any $V_F, N_P, V_P, \mu$, the function $\pi^v_F$ is either concave in $N_F$ or decreasing in $N_F$.
\end{lemma}
\begin{proof}
    Did not manage to prove it yet.
\end{proof}

\begin{corollary}
    \label{cor:unique_eq}
    There is at most one equilibrium with $N_F > 0$.
\end{corollary}

\begin{proposition}
    \label{prop:A_decreasing_in_N_P}
    The equilibrium aggregate size $A$ is monotonically decreasing in $N_P$ and $V_P$. When the fringe is feasible to begin with, the decrease is strict.
\end{proposition}
\begin{proof}
    From \cref{prop:fringe_profit_decreasing_in_N_P}, we know that the fringe profit is decreasing in $N_P$ and $V_P$.
    Furthermore, \cref{lem:shape_of_pi_F}
\end{proof}


\begin{corollary}
    \label{consumer_welfare_decreasing_in_N_P}
\end{corollary}
\begin{proof}
    
\end{proof}



\end{document}